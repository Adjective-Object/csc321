\documentclass[]{article}
\usepackage{amsmath}
\usepackage{blkarray}
\usepackage{pgfplots}
\usepackage{wrapfig}
\usepackage{placeins}
\usepackage{subfig}

\newcommand{\norm}[1]{\left\lVert#1\right\rVert}
\usepackage{tikz}
\usetikzlibrary{arrows,shapes}
\usepackage{multicol}
\usepackage{lmodern}
\usepackage{amssymb,amsmath}
\usepackage{ifxetex,ifluatex}
\usepackage{fixltx2e} % provides \textsubscript
\ifnum 0\ifxetex 1\fi\ifluatex 1\fi=0 % if pdftex
  \usepackage[T1]{fontenc}
  \usepackage[utf8]{inputenc}
\else % if luatex or xelatex
  \ifxetex
    \usepackage{mathspec}
    \usepackage{xltxtra,xunicode}
  \else
    \usepackage{fontspec}
  \fi
  \defaultfontfeatures{Mapping=tex-text,Scale=MatchLowercase}
  \newcommand{\euro}{€}
\fi
% use upquote if available, for straight quotes in verbatim environments
\IfFileExists{upquote.sty}{\usepackage{upquote}}{}
% use microtype if available
\IfFileExists{microtype.sty}{%
\usepackage{microtype}
\UseMicrotypeSet[protrusion]{basicmath} % disable protrusion for tt fonts
}{}
\usepackage[margin=1in]{geometry}
\ifxetex
  \usepackage[setpagesize=false, % page size defined by xetex
              unicode=false, % unicode breaks when used with xetex
              xetex]{hyperref}
\else
  \usepackage[unicode=true]{hyperref}
\fi
\hypersetup{breaklinks=true,
            bookmarks=true,
            pdfauthor={Maxwell Huang-Hobbs (g4rbage)},
            pdftitle={CSC321 A2 ; classifying handwritten digits using single and muti-layer neural networks},
            colorlinks=true,
            citecolor=blue,
            urlcolor=blue,
            linkcolor=magenta,
            pdfborder={0 0 0}}
\urlstyle{same}  % don't use monospace font for urls
\usepackage{color}
\usepackage{fancyvrb}
\newcommand{\VerbBar}{|}
\newcommand{\VERB}{\Verb[commandchars=\\\{\}]}
\DefineVerbatimEnvironment{Highlighting}{Verbatim}{commandchars=\\\{\}}
% Add ',fontsize=\small' for more characters per line
\newenvironment{Shaded}{}{}
\newcommand{\KeywordTok}[1]{\textcolor[rgb]{0.00,0.44,0.13}{\textbf{{#1}}}}
\newcommand{\DataTypeTok}[1]{\textcolor[rgb]{0.56,0.13,0.00}{{#1}}}
\newcommand{\DecValTok}[1]{\textcolor[rgb]{0.25,0.63,0.44}{{#1}}}
\newcommand{\BaseNTok}[1]{\textcolor[rgb]{0.25,0.63,0.44}{{#1}}}
\newcommand{\FloatTok}[1]{\textcolor[rgb]{0.25,0.63,0.44}{{#1}}}
\newcommand{\ConstantTok}[1]{\textcolor[rgb]{0.53,0.00,0.00}{{#1}}}
\newcommand{\CharTok}[1]{\textcolor[rgb]{0.25,0.44,0.63}{{#1}}}
\newcommand{\SpecialCharTok}[1]{\textcolor[rgb]{0.25,0.44,0.63}{{#1}}}
\newcommand{\StringTok}[1]{\textcolor[rgb]{0.25,0.44,0.63}{{#1}}}
\newcommand{\VerbatimStringTok}[1]{\textcolor[rgb]{0.25,0.44,0.63}{{#1}}}
\newcommand{\SpecialStringTok}[1]{\textcolor[rgb]{0.73,0.40,0.53}{{#1}}}
\newcommand{\ImportTok}[1]{{#1}}
\newcommand{\CommentTok}[1]{\textcolor[rgb]{0.38,0.63,0.69}{\textit{{#1}}}}
\newcommand{\DocumentationTok}[1]{\textcolor[rgb]{0.73,0.13,0.13}{\textit{{#1}}}}
\newcommand{\AnnotationTok}[1]{\textcolor[rgb]{0.38,0.63,0.69}{\textbf{\textit{{#1}}}}}
\newcommand{\CommentVarTok}[1]{\textcolor[rgb]{0.38,0.63,0.69}{\textbf{\textit{{#1}}}}}
\newcommand{\OtherTok}[1]{\textcolor[rgb]{0.00,0.44,0.13}{{#1}}}
\newcommand{\FunctionTok}[1]{\textcolor[rgb]{0.02,0.16,0.49}{{#1}}}
\newcommand{\VariableTok}[1]{\textcolor[rgb]{0.10,0.09,0.49}{{#1}}}
\newcommand{\ControlFlowTok}[1]{\textcolor[rgb]{0.00,0.44,0.13}{\textbf{{#1}}}}
\newcommand{\OperatorTok}[1]{\textcolor[rgb]{0.40,0.40,0.40}{{#1}}}
\newcommand{\BuiltInTok}[1]{{#1}}
\newcommand{\ExtensionTok}[1]{{#1}}
\newcommand{\PreprocessorTok}[1]{\textcolor[rgb]{0.74,0.48,0.00}{{#1}}}
\newcommand{\AttributeTok}[1]{\textcolor[rgb]{0.49,0.56,0.16}{{#1}}}
\newcommand{\RegionMarkerTok}[1]{{#1}}
\newcommand{\InformationTok}[1]{\textcolor[rgb]{0.38,0.63,0.69}{\textbf{\textit{{#1}}}}}
\newcommand{\WarningTok}[1]{\textcolor[rgb]{0.38,0.63,0.69}{\textbf{\textit{{#1}}}}}
\newcommand{\AlertTok}[1]{\textcolor[rgb]{1.00,0.00,0.00}{\textbf{{#1}}}}
\newcommand{\ErrorTok}[1]{\textcolor[rgb]{1.00,0.00,0.00}{\textbf{{#1}}}}
\newcommand{\NormalTok}[1]{{#1}}
\setlength{\parindent}{0pt}
\setlength{\parskip}{6pt plus 2pt minus 1pt}
\setlength{\emergencystretch}{3em}  % prevent overfull lines
\providecommand{\tightlist}{%
  \setlength{\itemsep}{0pt}\setlength{\parskip}{0pt}}
\setcounter{secnumdepth}{0}

\title{CSC321 A2 ; classifying handwritten digits using single and muti-layer
neural networks}
\author{Maxwell Huang-Hobbs (g4rbage)}
\date{}

% Redefines (sub)paragraphs to behave more like sections
\ifx\paragraph\undefined\else
\let\oldparagraph\paragraph
\renewcommand{\paragraph}[1]{\oldparagraph{#1}\mbox{}}
\fi
\ifx\subparagraph\undefined\else
\let\oldsubparagraph\subparagraph
\renewcommand{\subparagraph}[1]{\oldsubparagraph{#1}\mbox{}}
\fi
\begin{document}
\maketitle

\section{Part 1 - Description of
 Dataset}\label{part-1---description-of-dataset}

The dataset is the MNIST digit dataset, consisting of 1000 handwritten
characters of each of {[}0,1,2,3,4,5,6,7,8,9{]} represented as 28x28
grayscale images.

\begin{figure}[!h]
	\centering
	\captionsetup[subfigure]{labelformat=empty}
	\begin{tabular}{cccccccccc}
		\subfloat[]{\includegraphics[width = 0.25in]{part1/0_0.png}} &   
		\subfloat[]{\includegraphics[width = 0.25in]{part1/0_1.png}} &   
		\subfloat[]{\includegraphics[width = 0.25in]{part1/0_2.png}} &   
		\subfloat[]{\includegraphics[width = 0.25in]{part1/0_3.png}} &   
		\subfloat[]{\includegraphics[width = 0.25in]{part1/0_4.png}} &   
		\subfloat[]{\includegraphics[width = 0.25in]{part1/0_5.png}} &   
		\subfloat[]{\includegraphics[width = 0.25in]{part1/0_6.png}} &   
		\subfloat[]{\includegraphics[width = 0.25in]{part1/0_7.png}} &   
		\subfloat[]{\includegraphics[width = 0.25in]{part1/0_8.png}} &   
		\subfloat[]{\includegraphics[width = 0.25in]{part1/0_9.png}} \\
		\subfloat[]{\includegraphics[width = 0.25in]{part1/1_0.png}} &   
		\subfloat[]{\includegraphics[width = 0.25in]{part1/1_1.png}} &   
		\subfloat[]{\includegraphics[width = 0.25in]{part1/1_2.png}} &   
		\subfloat[]{\includegraphics[width = 0.25in]{part1/1_3.png}} &   
		\subfloat[]{\includegraphics[width = 0.25in]{part1/1_4.png}} &   
		\subfloat[]{\includegraphics[width = 0.25in]{part1/1_5.png}} &   
		\subfloat[]{\includegraphics[width = 0.25in]{part1/1_6.png}} &   
		\subfloat[]{\includegraphics[width = 0.25in]{part1/1_7.png}} &   
		\subfloat[]{\includegraphics[width = 0.25in]{part1/1_8.png}} &   
		\subfloat[]{\includegraphics[width = 0.25in]{part1/1_9.png}} \\
		\subfloat[]{\includegraphics[width = 0.25in]{part1/2_0.png}} &   
		\subfloat[]{\includegraphics[width = 0.25in]{part1/2_1.png}} &   
		\subfloat[]{\includegraphics[width = 0.25in]{part1/2_2.png}} &   
		\subfloat[]{\includegraphics[width = 0.25in]{part1/2_3.png}} &   
		\subfloat[]{\includegraphics[width = 0.25in]{part1/2_4.png}} &   
		\subfloat[]{\includegraphics[width = 0.25in]{part1/2_5.png}} &   
		\subfloat[]{\includegraphics[width = 0.25in]{part1/2_6.png}} &   
		\subfloat[]{\includegraphics[width = 0.25in]{part1/2_7.png}} &   
		\subfloat[]{\includegraphics[width = 0.25in]{part1/2_8.png}} &   
		\subfloat[]{\includegraphics[width = 0.25in]{part1/2_9.png}} \\
		\subfloat[]{\includegraphics[width = 0.25in]{part1/3_0.png}} &   
		\subfloat[]{\includegraphics[width = 0.25in]{part1/3_1.png}} &   
		\subfloat[]{\includegraphics[width = 0.25in]{part1/3_2.png}} &   
		\subfloat[]{\includegraphics[width = 0.25in]{part1/3_3.png}} &   
		\subfloat[]{\includegraphics[width = 0.25in]{part1/3_4.png}} &   
		\subfloat[]{\includegraphics[width = 0.25in]{part1/3_5.png}} &   
		\subfloat[]{\includegraphics[width = 0.25in]{part1/3_6.png}} &   
		\subfloat[]{\includegraphics[width = 0.25in]{part1/3_7.png}} &   
		\subfloat[]{\includegraphics[width = 0.25in]{part1/3_8.png}} &   
		\subfloat[]{\includegraphics[width = 0.25in]{part1/3_9.png}} \\
		\subfloat[]{\includegraphics[width = 0.25in]{part1/4_0.png}} &   
		\subfloat[]{\includegraphics[width = 0.25in]{part1/4_1.png}} &   
		\subfloat[]{\includegraphics[width = 0.25in]{part1/4_2.png}} &   
		\subfloat[]{\includegraphics[width = 0.25in]{part1/4_3.png}} &   
		\subfloat[]{\includegraphics[width = 0.25in]{part1/4_4.png}} &   
		\subfloat[]{\includegraphics[width = 0.25in]{part1/4_5.png}} &   
		\subfloat[]{\includegraphics[width = 0.25in]{part1/4_6.png}} &   
		\subfloat[]{\includegraphics[width = 0.25in]{part1/4_7.png}} &   
		\subfloat[]{\includegraphics[width = 0.25in]{part1/4_8.png}} &   
		\subfloat[]{\includegraphics[width = 0.25in]{part1/4_9.png}} \\
		\subfloat[]{\includegraphics[width = 0.25in]{part1/5_0.png}} &   
		\subfloat[]{\includegraphics[width = 0.25in]{part1/5_1.png}} &   
		\subfloat[]{\includegraphics[width = 0.25in]{part1/5_2.png}} &   
		\subfloat[]{\includegraphics[width = 0.25in]{part1/5_3.png}} &   
		\subfloat[]{\includegraphics[width = 0.25in]{part1/5_4.png}} &   
		\subfloat[]{\includegraphics[width = 0.25in]{part1/5_5.png}} &   
		\subfloat[]{\includegraphics[width = 0.25in]{part1/5_6.png}} &   
		\subfloat[]{\includegraphics[width = 0.25in]{part1/5_7.png}} &   
		\subfloat[]{\includegraphics[width = 0.25in]{part1/5_8.png}} &   
		\subfloat[]{\includegraphics[width = 0.25in]{part1/5_9.png}} \\
		\subfloat[]{\includegraphics[width = 0.25in]{part1/1_0.png}} &   
		\subfloat[]{\includegraphics[width = 0.25in]{part1/6_1.png}} &   
		\subfloat[]{\includegraphics[width = 0.25in]{part1/6_2.png}} &   
		\subfloat[]{\includegraphics[width = 0.25in]{part1/6_3.png}} &   
		\subfloat[]{\includegraphics[width = 0.25in]{part1/6_4.png}} &   
		\subfloat[]{\includegraphics[width = 0.25in]{part1/6_5.png}} &   
		\subfloat[]{\includegraphics[width = 0.25in]{part1/6_6.png}} &   
		\subfloat[]{\includegraphics[width = 0.25in]{part1/6_7.png}} &   
		\subfloat[]{\includegraphics[width = 0.25in]{part1/6_8.png}} &   
		\subfloat[]{\includegraphics[width = 0.25in]{part1/6_9.png}} \\
		\subfloat[]{\includegraphics[width = 0.25in]{part1/7_0.png}} &   
		\subfloat[]{\includegraphics[width = 0.25in]{part1/7_1.png}} &   
		\subfloat[]{\includegraphics[width = 0.25in]{part1/7_2.png}} &   
		\subfloat[]{\includegraphics[width = 0.25in]{part1/7_3.png}} &   
		\subfloat[]{\includegraphics[width = 0.25in]{part1/7_4.png}} &   
		\subfloat[]{\includegraphics[width = 0.25in]{part1/7_5.png}} &   
		\subfloat[]{\includegraphics[width = 0.25in]{part1/7_6.png}} &   
		\subfloat[]{\includegraphics[width = 0.25in]{part1/7_7.png}} &   
		\subfloat[]{\includegraphics[width = 0.25in]{part1/7_8.png}} &   
		\subfloat[]{\includegraphics[width = 0.25in]{part1/7_9.png}} \\
		\subfloat[]{\includegraphics[width = 0.25in]{part1/8_0.png}} &   
		\subfloat[]{\includegraphics[width = 0.25in]{part1/8_1.png}} &   
		\subfloat[]{\includegraphics[width = 0.25in]{part1/8_2.png}} &   
		\subfloat[]{\includegraphics[width = 0.25in]{part1/8_3.png}} &   
		\subfloat[]{\includegraphics[width = 0.25in]{part1/8_4.png}} &   
		\subfloat[]{\includegraphics[width = 0.25in]{part1/8_5.png}} &   
		\subfloat[]{\includegraphics[width = 0.25in]{part1/8_6.png}} &   
		\subfloat[]{\includegraphics[width = 0.25in]{part1/8_7.png}} &   
		\subfloat[]{\includegraphics[width = 0.25in]{part1/8_8.png}} &   
		\subfloat[]{\includegraphics[width = 0.25in]{part1/8_9.png}} \\
		\subfloat[]{\includegraphics[width = 0.25in]{part1/9_0.png}} &   
		\subfloat[]{\includegraphics[width = 0.25in]{part1/9_1.png}} &   
		\subfloat[]{\includegraphics[width = 0.25in]{part1/9_2.png}} &   
		\subfloat[]{\includegraphics[width = 0.25in]{part1/9_3.png}} &   
		\subfloat[]{\includegraphics[width = 0.25in]{part1/9_4.png}} &   
		\subfloat[]{\includegraphics[width = 0.25in]{part1/9_5.png}} &   
		\subfloat[]{\includegraphics[width = 0.25in]{part1/9_6.png}} &   
		\subfloat[]{\includegraphics[width = 0.25in]{part1/9_7.png}} &   
		\subfloat[]{\includegraphics[width = 0.25in]{part1/9_8.png}} &   
		\subfloat[]{\includegraphics[width = 0.25in]{part1/9_9.png}} \\
	\end{tabular}
	\caption{Examples of input data}
\end{figure}

\section{Part 2 - Implementing the
 network}\label{part-2---implementing-the-network}

\textbf{Not: The first half of this assignment was implemented by
	concatenating biases to the front of the weight matrix, concatenating a
row of constants to the front of the input matrix}

\textbf{Additionally, N was used as the first matrix index}

The network is implemented with the following function:

\begin{Shaded}
	\begin{Highlighting}[]
		\KeywordTok{def} \NormalTok{linear_network(x, W):}
		\ControlFlowTok{return} \NormalTok{softmax(dot(x, W))}
	\end{Highlighting}
\end{Shaded}

\section{Part 3 - Gradient of the Single Layer
 Network}\label{part-3---gradient-of-the-single-layer-network}

The gradient of the network was implemented using the following function

\begin{Shaded}
	\begin{Highlighting}[]
		\KeywordTok{def} \NormalTok{grad_neg_log_likelihood(inp, weights, targets):}
		\NormalTok{N, _ }\OperatorTok{=} \NormalTok{inp.shape}
		\NormalTok{I, O }\OperatorTok{=} \NormalTok{weights.shape}
		
		\NormalTok{probability }\OperatorTok{=} \NormalTok{softmax(dot(inp.reshape((N, I)), weights)}
		
		\OperatorTok{=} \NormalTok{(output)}
		\NormalTok{pdiffs }\OperatorTok{=} \NormalTok{(probability }\OperatorTok{-} \NormalTok{targets)}
		\NormalTok{pdiffs }\OperatorTok{=} \NormalTok{tile(pdiffs.reshape((N, }\DecValTok{1}\NormalTok{, O)), (}\DecValTok{1}\NormalTok{, I, }\DecValTok{1}\NormalTok{))}
		\NormalTok{inp_expanded }\OperatorTok{=} \NormalTok{inp.reshape((N, I, }\DecValTok{1}\NormalTok{))}
		\NormalTok{inp_expanded }\OperatorTok{=} \NormalTok{tile(inp_expanded, (}\DecValTok{1}\NormalTok{, }\DecValTok{1}\NormalTok{, O))}
		
		\ControlFlowTok{return} \NormalTok{pdiffs }\OperatorTok{*} \NormalTok{inp_expanded}
	\end{Highlighting}
\end{Shaded}

\clearpage
\newpage

\section{Part 4 - Verifying the Gradient of the Single Layer
 network.}\label{part-4---verifying-the-gradient-of-the-single-layer-network.}

\begin{wrapfigure}[13]{r}{0.5\textwidth}
	\vspace{-20pt}
	\centering
	\includegraphics[width=0.4\textwidth]{part4/difference_histogram.png}
	\caption{ difference $(grad_{approx} - grad_{calc})$ between the approximate and calculated gradient for the linear network}
\end{wrapfigure}

The gradient of the network was approximated by changing the weight of
each value in the weight matrix by a small step value (0.01) upwards and
downwards, and calculating the slope of the cost over that difference

The difference between corresponding values in the approximate and
calculated gradients is highly centered around 0, which would suggest
that the gradient function is accurate.

The large tails on either side of the distribution could be explained by
the approximation function stepping over abrupt changes in the cost
function. This could be corrected for by using a smaller step function
for approximating the gradient.

\section{Part 5 - Gradient Descent with the Single Layer
 Network}\label{part-5---gradient-descent-with-the-single-layer-network}

\begin{figure}[!h]
	\vspace{-20pt}
	\centering
	\includegraphics[width=0.8\textwidth]{part5/learning_rate.png}
	\caption{Learning rate for the single layer neural network versus generations of the network}
\end{figure}

The neural network was trained on the training set with the gradient
function from Part 4 using the batch size of 50 samples from each digit
(500 total), and a constant learning rate of 0.01.

The final neural network performed with \(96.1\)\% accuracy on the
testing set.

\begin{figure}[!h]
	\centering
	\captionsetup[subfigure]{labelformat=empty}
	\begin{tabular}{cccccccccc}
		\subfloat[]{\includegraphics[width = 0.25in]{part5/succ_0.png}}  &   
		\subfloat[]{\includegraphics[width = 0.25in]{part5/succ_1.png}}  &   
		\subfloat[]{\includegraphics[width = 0.25in]{part5/succ_2.png}}  &   
		\subfloat[]{\includegraphics[width = 0.25in]{part5/succ_3.png}}  &   
		\subfloat[]{\includegraphics[width = 0.25in]{part5/succ_4.png}}  &   
		\subfloat[]{\includegraphics[width = 0.25in]{part5/succ_5.png}}  &   
		\subfloat[]{\includegraphics[width = 0.25in]{part5/succ_6.png}}  &   
		\subfloat[]{\includegraphics[width = 0.25in]{part5/succ_7.png}}  &   
		\subfloat[]{\includegraphics[width = 0.25in]{part5/succ_8.png}}  &   
		\subfloat[]{\includegraphics[width = 0.25in]{part5/succ_9.png}} \\
		\subfloat[]{\includegraphics[width = 0.25in]{part5/succ_10.png}} &   
		\subfloat[]{\includegraphics[width = 0.25in]{part5/succ_11.png}} &   
		\subfloat[]{\includegraphics[width = 0.25in]{part5/succ_12.png}} &   
		\subfloat[]{\includegraphics[width = 0.25in]{part5/succ_13.png}} &   
		\subfloat[]{\includegraphics[width = 0.25in]{part5/succ_14.png}} &   
		\subfloat[]{\includegraphics[width = 0.25in]{part5/succ_15.png}} &   
		\subfloat[]{\includegraphics[width = 0.25in]{part5/succ_16.png}} &   
		\subfloat[]{\includegraphics[width = 0.25in]{part5/succ_17.png}} &   
		\subfloat[]{\includegraphics[width = 0.25in]{part5/succ_18.png}} &   
		\subfloat[]{\includegraphics[width = 0.25in]{part5/succ_19.png}} \\
	\end{tabular}
	\caption{Examples of images the network correctly classifies}
\end{figure}

\begin{figure}[!h]
	\centering
	\captionsetup[subfigure]{labelformat=empty}
	\begin{tabular}{cccccccccc}
		\subfloat[]{\includegraphics[width = 0.25in]{part5/failure_0.png}} &   
		\subfloat[]{\includegraphics[width = 0.25in]{part5/failure_1.png}} &   
		\subfloat[]{\includegraphics[width = 0.25in]{part5/failure_2.png}} &   
		\subfloat[]{\includegraphics[width = 0.25in]{part5/failure_3.png}} &   
		\subfloat[]{\includegraphics[width = 0.25in]{part5/failure_4.png}} &   
		\subfloat[]{\includegraphics[width = 0.25in]{part5/failure_5.png}} &   
		\subfloat[]{\includegraphics[width = 0.25in]{part5/failure_6.png}} &   
		\subfloat[]{\includegraphics[width = 0.25in]{part5/failure_7.png}} &   
		\subfloat[]{\includegraphics[width = 0.25in]{part5/failure_8.png}} &   
		\subfloat[]{\includegraphics[width = 0.25in]{part5/failure_9.png}} \\
	\end{tabular}
	\caption{Examples of images the network fails to  classify}
\end{figure}

The images the neural network failed to correctly classify are mostly
either drawn with thin strokes, or skewed in some direction.

Failure to classify digits with various stroke widths can be explained
by the fact that the neural network directly maps each intensity on the
input layer to some output on the output layer, so a thinly stroked
digit would have low intensity in many of the highly-weighted outputs.

Failure to classify skewed digits could likewise be explained by the
structure of the neural network.

\section{Part 6 - Visualizing the weights of the inputs of the neural
 network}\label{part-6---visualizing-the-weights-of-the-inputs-of-the-neural-network}

The weights of the inputs to the neural network look like blurred
versions of their corresponding digits. They also seem to have some
negative noise around the outside of the borders of each digit, which is
likely weighting against the other digits.

\begin{figure}[!h]
	\centering
	\captionsetup[subfigure]{labelformat=empty}
	\begin{tabular}{cccccccccc}
		\subfloat[]{\includegraphics[width = 1in]{part6/part6_0.png}} &   
		\subfloat[]{\includegraphics[width = 1in]{part6/part6_1.png}} &   
		\subfloat[]{\includegraphics[width = 1in]{part6/part6_2.png}} &   
		\subfloat[]{\includegraphics[width = 1in]{part6/part6_3.png}} &   
		\subfloat[]{\includegraphics[width = 1in]{part6/part6_4.png}} \\
		\subfloat[]{\includegraphics[width = 1in]{part6/part6_5.png}} &   
		\subfloat[]{\includegraphics[width = 1in]{part6/part6_6.png}} &   
		\subfloat[]{\includegraphics[width = 1in]{part6/part6_7.png}} &   
		\subfloat[]{\includegraphics[width = 1in]{part6/part6_8.png}} &   
		\subfloat[]{\includegraphics[width = 1in]{part6/part6_9.png}} \\
	\end{tabular}
	\caption{input weights of the neural network}
\end{figure}

\clearpage

\section{Part 7 - Implementation \& Explanation of the Gradient
 Function}\label{part-7---implementation-explanation-of-the-gradient-function}

The following is the implementation of the gradient function used in
this project. Some matrix reshaping at the beginning is omitted
(accounting for the shape of the data)

\begin{Shaded}
	\begin{Highlighting}[]
		\KeywordTok{def} \NormalTok{dtanh(y):}
		\ControlFlowTok{return} \FloatTok{1.0} \OperatorTok{-} \NormalTok{(y}\OperatorTok{**} \DecValTok{2}\NormalTok{)}
		
		\KeywordTok{def} \NormalTok{grad_multilayer((W0, b0), (W1, b1), inp, expected_output):}
		\CommentTok{# convenience variables}
		\NormalTok{I, N }\OperatorTok{=} \NormalTok{inp.shape}
		\NormalTok{H, O }\OperatorTok{=} \NormalTok{W1.shape}
		
		\CommentTok{# run through the neural network}
		\NormalTok{L0, L1, prediction }\OperatorTok{=} \NormalTok{forward(inp, (W0, b0), (W1,b1))}
		
		\CommentTok{# partial derivatives }
		\NormalTok{dCdL1 }\OperatorTok{=} \NormalTok{(prediction }\OperatorTok{-} \NormalTok{expected_output)}
		\NormalTok{dL1dL0 }\OperatorTok{=} \NormalTok{dtanh(W1)}
		\NormalTok{dCdL0 }\OperatorTok{=} \NormalTok{dot(dL1dL0, dCdL1)}
		
		\CommentTok{# derivative of layers}
		\NormalTok{dL1dW1 }\OperatorTok{=} \NormalTok{dtanh(L0)  }\CommentTok{# L1 = tanh(dot(W1, L0))}
		\NormalTok{dL0dW0 }\OperatorTok{=} \NormalTok{dtanh(inp) }\CommentTok{# L0 = tanh(dot(W0, inp)}
		
		\NormalTok{reshape }\OperatorTok{and} \NormalTok{do a scalar multiplication instead of tiling }\OperatorTok{and} \NormalTok{doing a dot product to avoid issues}
		\NormalTok{gradients_W1 }\OperatorTok{=} \NormalTok{dCdL1.reshape((}\DecValTok{1}\NormalTok{, O, N)) }\OperatorTok{*} \NormalTok{dL1dW1.reshape((H, }\DecValTok{1}\NormalTok{, N))}
		\NormalTok{gradients_W0 }\OperatorTok{=} \NormalTok{dCdL0.reshape((}\DecValTok{1}\NormalTok{, H, N)) }\OperatorTok{*} \NormalTok{dL0dW0.reshape((I, }\DecValTok{1}\NormalTok{, N))}
		
		\ControlFlowTok{return} \NormalTok{(gradients_W0, dCdL0), (gradients_W1, dCdL1), prediction}
	\end{Highlighting}
\end{Shaded}

\subsubsection{Partial Derivatives}\label{partial-derivatives}

\begin{enumerate}
	\def\labelenumi{\arabic{enumi}.}
	\item
	      \begin{verbatim}
dCdL1 = (prediction - expected_output)
	\end{verbatim}
	
	\(\dfrac{\delta C}{\delta L_1} = \text{prediction} - \text{expected\_output}\)
	given in the slides on one-hot encoding
	\item
	      \begin{verbatim}
dL1dL0 = dtanh(W1)
	\end{verbatim}
	
	Let \(M\) be the output of the linear layer
	\(L_0 \rightarrow W_1 \rightarrow M\)
	
	\(\dfrac{\delta L_1^i}{\delta L_0^j} =  \dfrac{\delta L_1^i}{\delta M}  \dfrac{\delta M}{\delta L0}\)
	
	\(\dfrac{\delta M_1^i}{\delta L_0^j} M_1^i =  \dfrac{\delta M_1^i}{\delta L_0^j} \sum_{J} W_1^{i,J} L_0^j =  (W_1^{i,0} * 0 + .. + W_1^{i,j} + .. + W_1^{i,300} * 0) =  W_1^{i,j}\)
	
	\(\dfrac{\delta L_1^i}{\delta M_1^i} =  \dfrac{\delta}{\delta M_1^i} tanh(M_1^i) = dtanh(M_1^i)\)
	
	Combining these partial derivatives we get,
	
	\(\dfrac{\delta L_1^i}{\delta L_0^j} =  dtanh(M_1^i) W_1^{i,j}\)
	
	TODO FIX IN CODE
	\item
	      \begin{verbatim}
dCdL0 = dot(dL1dL0, dCdL1)
	\end{verbatim}
	
	By the chain rule, and parts (1.) and (2.)
\end{enumerate}

\subsubsection{Derivatives of Layers}\label{derivatives-of-layers}

\begin{enumerate}
	\def\labelenumi{\arabic{enumi}.}
	\item
	      \begin{verbatim}
dL1dW1 = dtanh(L0)
	\end{verbatim}
	
	\(L_1^i = tanh(\sum_J(W_1^{i,j}, L_0^j))\)
	\(\cfrac{\delta}{\delta W_1^{i,j}} L_1^i =  \cfrac{\delta L_1^i}{\delta M^i}  \cfrac{\delta M^i}{\delta W_1^{i,j}}\)
	
	\(\cfrac{\delta L_1^i }{\delta W_1^{i,j}}=  dtanh(M^i)  \cfrac{\delta M^i}{\delta W_1^{i,j}}  \sum_J W_1^{i,J} * L_0^J\)
	
	\(\cfrac{\delta L_1^i }{\delta W_1^{i,j}}=  dtanh(M^i)  (0 * L_0^0 + .. +L_0^j + .. + 0 * L_0^300)\)
	
	\(\cfrac{\delta L_1^i}{\delta W_1^{i,j}} =  dtanh(M^i) L_0^j\)
	
	TODO fix in code
	\item
	      \begin{verbatim}
dL0dW0 = dtanh(inp)
	\end{verbatim}
	
	By the same process as (1.),
	
	\(\cfrac{\delta L_0^i}{\delta W_0^{i,j}} =  dtanh(M) inp^j\)
	
	TODO fix in code
	\item
	      \begin{verbatim}
gradients_W1 = dCdL1.reshape((1, O, N)) * dL1dW1.reshape((H, 1, N))
gradients_W0 = dCdL0.reshape((1, H, N)) * dL0dW0.reshape((I, 1, N))
	\end{verbatim}
	
	By the chain rule,
	\(\cfrac{C}{\delta W_1} = \cfrac{\delta C}{\delta L_1} \cfrac{\delta L_1}{\delta W_1}\)
	
	\(\cfrac{C}{\delta W_0} = \cfrac{\delta C}{\delta L_0} \cfrac{\delta L_0}{\delta W_0}\)
	
	A multiplication between matrices of mismatched dimensions broadcasts
	the arrays accross each other, which is equivalent to tiling and doing
	a dot product between the arrays
	
	TODO does this even work?
\end{enumerate}

\clearpage

\section{Part 8 - Approximation to the
 Gradient}\label{part-8---approximation-to-the-gradient}

The difference between the actual and expected gradients was mostly
promising (centered around 0, low spread). However, the implementation
of the was likely inaccurate in some areas, as the distribution of the
gradients of the offset matrix \(b_1\) was unpatterned and widely
spread.

\begin{figure}[!h]
	\centering
	\captionsetup[subfigure]{labelformat=empty}
	\begin{tabular}{cc}
		\subfloat[]{\includegraphics[height = 2in]{part8/difference_histogram_W0}} &   
		\subfloat[]{\includegraphics[height = 2in]{part8/difference_histogram_W1}} \\
		\subfloat[]{\includegraphics[height = 2in]{part8/difference_histogram_b0}} &   
		\subfloat[]{\includegraphics[height = 2in]{part8/difference_histogram_b1}} \\
	\end{tabular}
	\caption{difference approximated - calculated for all values in the gradients of each of $W_0$ (top left), $W_1$ (top right), $b_0$(bottom left), and $b_1$(bottom right)}
\end{figure}

\begin{wrapfigure}[13]{r}{0.5\textwidth}
	\vspace{-40pt}
	\centering
	\includegraphics[width=0.4\textwidth]{part9/multilayer_learning_curve.png}
	\caption{ Learning curve of the multilayer neural network }
\end{wrapfigure}

\section{Part 9 - Training with the multilayer
 gradient}\label{part-9---training-with-the-multilayer-gradient}

As one would expect with a partially incorrect implementation of the
gradient function, the neural network did not perform well.

It hovers briefly around \(44\)\% accuracy, before quickly falling off
to around \(10\)\% accuracy. This would suggest that the network is
memorizing the features of only one of the classes of digits, and always
reporting the digit to b that output

\clearpage

\begin{figure}[!h]
	\centering
	\captionsetup[subfigure]{labelformat=empty}
	\begin{tabular}{cccccccccc}
		\subfloat[]{\includegraphics[width = 0.25in]{part9/multilayer_succ_0.png}}  &   
		\subfloat[]{\includegraphics[width = 0.25in]{part9/multilayer_succ_1.png}}  &   
		\subfloat[]{\includegraphics[width = 0.25in]{part9/multilayer_succ_2.png}}  &   
		\subfloat[]{\includegraphics[width = 0.25in]{part9/multilayer_succ_3.png}}  &   
		\subfloat[]{\includegraphics[width = 0.25in]{part9/multilayer_succ_4.png}}  &   
		\subfloat[]{\includegraphics[width = 0.25in]{part9/multilayer_succ_5.png}}  &   
		\subfloat[]{\includegraphics[width = 0.25in]{part9/multilayer_succ_6.png}}  &   
		\subfloat[]{\includegraphics[width = 0.25in]{part9/multilayer_succ_7.png}}  &   
		\subfloat[]{\includegraphics[width = 0.25in]{part9/multilayer_succ_8.png}}  &   
		\subfloat[]{\includegraphics[width = 0.25in]{part9/multilayer_succ_9.png}} \\
		\subfloat[]{\includegraphics[width = 0.25in]{part9/multilayer_succ_10.png}} &   
		\subfloat[]{\includegraphics[width = 0.25in]{part9/multilayer_succ_11.png}} &   
		\subfloat[]{\includegraphics[width = 0.25in]{part9/multilayer_succ_12.png}} &   
		\subfloat[]{\includegraphics[width = 0.25in]{part9/multilayer_succ_13.png}} &   
		\subfloat[]{\includegraphics[width = 0.25in]{part9/multilayer_succ_14.png}} &   
		\subfloat[]{\includegraphics[width = 0.25in]{part9/multilayer_succ_15.png}} &   
		\subfloat[]{\includegraphics[width = 0.25in]{part9/multilayer_succ_16.png}} &   
		\subfloat[]{\includegraphics[width = 0.25in]{part9/multilayer_succ_17.png}} &   
		\subfloat[]{\includegraphics[width = 0.25in]{part9/multilayer_succ_18.png}} &   
		\subfloat[]{\includegraphics[width = 0.25in]{part9/multilayer_succ_19.png}} \\
	\end{tabular}
	\caption{Examples of images the network fails to  classify}
\end{figure}

\begin{figure}[!h]
	\centering
	\captionsetup[subfigure]{labelformat=empty}
	\begin{tabular}{cccccccccc}
		\subfloat[]{\includegraphics[width = 0.25in]{part9/multilayer_failure_0.png}} &   
		\subfloat[]{\includegraphics[width = 0.25in]{part9/multilayer_failure_1.png}} &   
		\subfloat[]{\includegraphics[width = 0.25in]{part9/multilayer_failure_2.png}} &   
		\subfloat[]{\includegraphics[width = 0.25in]{part9/multilayer_failure_3.png}} &   
		\subfloat[]{\includegraphics[width = 0.25in]{part9/multilayer_failure_4.png}} &   
		\subfloat[]{\includegraphics[width = 0.25in]{part9/multilayer_failure_5.png}} &   
		\subfloat[]{\includegraphics[width = 0.25in]{part9/multilayer_failure_6.png}} &   
		\subfloat[]{\includegraphics[width = 0.25in]{part9/multilayer_failure_7.png}} &   
		\subfloat[]{\includegraphics[width = 0.25in]{part9/multilayer_failure_8.png}} &   
		\subfloat[]{\includegraphics[width = 0.25in]{part9/multilayer_failure_9.png}} \\
	\end{tabular}
	\caption{Examples of images the network fails to  classify}
\end{figure}

\section{Part 10 - Visualizing the input layer of the Neural
 Network}\label{part-10---visualizing-the-input-layer-of-the-neural-network}

The input layer to the multilayer neural network are mostly just random
noise, though some of them appear to be fitting to parts of different
digits

\begin{figure}[!h]
	\centering
	\captionsetup[subfigure]{labelformat=empty}
	\includegraphics[width = 1.5in]{part10/112.png}
	\caption{A slice of the first layer in the neural network}
\end{figure}

This slice of the input layer seems to be fitting to the character 2,
and against the space immediately around 2

\begin{figure}[!h]
	\centering
	\captionsetup[subfigure]{labelformat=empty}
	\includegraphics[width = 1.5in]{part10/114.png}
	\caption{Another slice of the first layer in the neural network}
\end{figure}

This slice of the input layer seems to be fitting the areas guaranteed
to be in the character 3, and against other areas of the image. The
break in the layer is possibly because that stroke in the character 3 is
often varied

\end{document}
